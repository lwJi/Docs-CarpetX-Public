\documentclass[prd,aps,a4paper,superscriptaddress,onecolumn,footinbib]{revtex4}
\usepackage{graphicx}
\usepackage{color}
\usepackage{dcolumn}
\usepackage{bm}
\usepackage{slashed}
\usepackage{amsmath}
\usepackage{latexsym}
\usepackage{amssymb}
\usepackage{amsfonts}
\usepackage{dsfont}
\usepackage{listings} 
\usepackage{xcolor} 
\usepackage{ulem}
\usepackage{cancel}
\usepackage{mathtools}

\begin{document}
\title{Radiative Boundary Conditions in CarpetX}

\author{Liwei Ji}
\email{ljsma@rit.edu}
\affiliation{Rochester Institute of Technology}

\maketitle

\tableofcontents

\lstset{numbers=left, 
	numberstyle= \tiny, 
	keywordstyle= \color{ blue!70},commentstyle=\color{red!50!green!50!blue!50}, 
	frame=shadowbox, 
	rulesepcolor= \color{ red!20!green!20!blue!20} 
} 	


\section{Scalar Waves}

EOM
\begin{align}
  \dot{u}
  &=\rho, \\
  \dot{\rho}
  &=\nabla \cdot \bm{v}, \\
  \dot{\bm{v}}
  &=\nabla \rho.
\end{align}
where characteristic matrix with respect to normal $\bm{n}$ can be writen as
\begin{align}
  A^{\bm{n}}=
\end{align}

Characteristics fields
\begin{align}
  \begin{array}{lll}
    u^{\hat{0}}    &=u, \quad                          &\text{speed} \quad 0, \\
    u^{\hat{1}\pm} &=\rho\mp\bm{n}\cdot\bm{v},         &\text{speed} \quad \pm 1, \\
    u^{\hat{2}}    &=\bm{v}-\bm{n}(\bm{n}\cdot\bm{v}), &\text{speed} \quad 0.
  \end{array}
\end{align}

Radiative boundary condition
\begin{align}
  u^{\hat{1}-}=0.
\end{align}









%\bibliography{refs}

\end{document}
